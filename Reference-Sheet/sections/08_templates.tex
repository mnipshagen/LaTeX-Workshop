\section{Reduce, Reuse, Recycle: Templating}

You can find a large variety of templates online: papers, slides, presentations, posters, CVs... 
The latex community has created different styles and layouts for all occasions. Overleaf hosts a lot of templates themselves and you can browse them on \url{https://www.overleaf.com/latex/templates}. A large part of journals also supply templates for latex that set-up the correct style, page layout and citation style, which makes it simpler to adhere to their guidelines.\\
Beyond those community-made ones, it might be worth it to create your own. For this reference sheet, you can find \texttt{basic\_setup.tex}. 
In this file we import some common packages, that we use in the majority of our projects and also set up some commands that we find useful. You can place this file somewhere on your computer and then start every latex document with\\ 
\texttt{\textbackslash input\{\{/path/\}\{to/\}\{your/\}\{setup\_template\}\}}\footnote{Notice, that there is no \texttt{.tex} at the end. You usually omit the file ending and latex picks the right file.}\\
This would set up your personal library of commands and packages that you like to use in every document.