While similar, there are important distinctions between both of them. Most of the time, you will probably be looking to use \textit{input}. With \textit{input} LaTeX takes the content of the second file and handles it as if you just copy-pasted it into the document. Hence your auxiliary file will need no preamble (e.g. document class) and you can reference definitions made in the file you are inputting. You can for example put your favourite packages into a file and input it at the top of every document.