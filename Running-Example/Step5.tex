\documentclass[12pt]{scrarticle}

\usepackage[utf8]{inputenc}
\usepackage{graphicx}
\usepackage{listings}
\usepackage[colorlinks=true]{hyperref}
\usepackage{amsmath}
\usepackage{lipsum}
\usepackage{caption}
\usepackage{subcaption}

\title{A running Example of Latex}
\author{Daniel Anthes \and Moritz Nipshagen}
\date{} % defaults to today

\begin{document}

\maketitle

\begin{abstract}
    \lipsum[3]
\end{abstract}

\section{I am a section}

\lipsum[1]

\subsection{I am a subsection}

\lipsum[2]

\section{Math}

Here is an inline formula: $e = mc^2$.

One more, but now on a separate line:

$$
\mathcal{N}(x|\mu, \sigma) = \frac{1}{\sigma \sqrt{2\pi}}\exp \left({\frac{(x - \mu)^2}{2\sigma^2}}\right)
$$

Now with multiple aligned formulas:

\begin{eqnarray}
\dot{x_1} &=& -x_2 + x_{2}^3\\
\dot{x_2} &=& -x_1 + x_{1}^3
\end{eqnarray}

Notice that \textit{eqnarray} numbers the formulas. Add a star to disable numbering:


\begin{eqnarray*}
\dot{x_1} &=& -x_2 + x_{2}^3\\
\dot{x_2} &=& -x_1 + x_{1}^3
\end{eqnarray*}

\section{Images}

\begin{figure}[h!]
    \centering
    \includegraphics[width=.3\linewidth]{Running-Example/zebra.jpeg}
    \caption{Zebra.}
    \label{fig:examplefig}
\end{figure}

We can also have multiple images in a figure:

\begin{figure}[h!]
    \centering
    \begin{subfigure}{0.49\textwidth}
        \includegraphics[width=0.9\linewidth]{Running-Example/zebra.jpeg}
        \caption{Zebra.}
    \end{subfigure}
    \begin{subfigure}{0.49\textwidth}
        \includegraphics[width=0.9\linewidth]{Running-Example/kangaroo.jpeg}
        \caption{Kangaro}
    \end{subfigure}
    \caption{Multiple images in a figure}
    \label{fig:texmeme}
\end{figure}

\newpage

\section{Tables}

\begin{table}[h!]
    \centering
    \begin{tabular}{|l||c|c|c|}
    \hline 
    Animal & Is Zebra & Is Pet & Legs\\
    \hline\hline
    Dog         & No    & Yes   & 4\\
    Cat         & No    & Yes   & 4\\
    Zebra       & Yes   & No    & 4\\
    Kangaroo    & No    & No    & 2\\
    \hline
    \end{tabular}
    \caption{Animal Facts}
    \label{tab:my_label}
\end{table}

\end{document}