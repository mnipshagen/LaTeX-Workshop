\documentclass[aspectratio=169]{beamer}
\usetheme{donders}

\usepackage[utf8]{inputenc}
\usepackage[T1]{fontenc}

\title{A \LaTeX workshop}
\date{03-05-2021}
\author{Moritz Nipshagen \and Daniel Anthes}


\begin{document}

\begin{frame}
\titlepage
\end{frame}

\begin{frame}{What Today is About}
\begin{enumerate} 
\item<1-> What is \LaTeX
\item<2-> Why \LaTeX
\item<3-> An Example
\item<4-> Our Overview Document
\item<5-> DIY: Write a Piece
\end{enumerate}
\end{frame}

\begin{frame}{What is \LaTeX}
\begin{itemize}
\item A "content first" document creator using plain text
\item A programmatic approach to document creation \& lay-outing
\item Simple to maintain, share and collaborate without being bound to a specific software
\item Free
\end{itemize}
\end{frame}

\begin{frame}{Why \LaTeX}
    \begin{itemize}
        \item all plain text:
        \begin{itemize}
            \item You can use git, etc. for versioning
            \item Cross-platform: Everyone can read and edit it with a bunch of tools
            \item Don't worry about lay-outing and optics while writing
        \end{itemize}
        \item Loads of templates and open-source material
        \item Change design and layout without modifying anything in your actual text
    \end{itemize}
\end{frame}

\begin{frame}{Why not \LaTeX}
    \begin{itemize}
        \item Not a WYSIWYG editor
        \item Hence, a steeper learning curve than e.g. Word
        \item With great control, come great problems
        \begin{itemize}
            \item Error messages are sometimes hard to understand without background knowledge
            \item Not all solutions are intuitive in a visual manner
        \end{itemize}
    \end{itemize}
    But not too worry, a lot of these problems are alleviated by modern solutions (e.g. overleaf) and the community (great help forums) 
\end{frame}

\begin{frame}{An Example}
    \textit{A live overview}
\end{frame}

\begin{frame}{Our Overview Document}
    \begin{itemize}
        \item<1-> There are great online resources
        \begin{itemize}
            \item The overleaf docs: \url{https://www.overleaf.com/learn}
            \item WikiBooks: \url{https://en.wikibooks.org/wiki/LaTeX}
            \item Tex Stackexchange: \url{https://tex.stackexchange.com/}
            \item and many blogs, tutorials, official documentation, etc.
        \end{itemize}
        \item<2-> We also wrote a tex document that contains and demonstrates, what we found to be common things to use
        \item<2-> You can use it as a reference sheet to look things up, to copy, paste \& adapt and as a template if you want
        \item<3-> Let's quickly go over it
    \end{itemize}
\end{frame}

\begin{frame}{Your Turn}
    In our opinion, \LaTeX is learned best by doing it yourself. Hence, the rest of the workshop is going to be time for you to write your own little essay and if you have questions we are here to help.
\end{frame}

\begin{frame}{Your Turn}
    In our opinion, \LaTeX is learned best by doing it yourself. Hence, the rest of the workshop is going to be time for you to write your own little essay and if you have questions we are here to help.\\
    The essay topic is: Your favourite animal\\
    If you want, you can send us the project when you are done via mail (or fork and send a pull request to the github repo) and we are going to make a small journal-esque collection of them:\\
    \begin{center}
        \textit{\huge DJUNGLE}\\
        \textbf{D}ondrite's \textbf{J}o\textbf{u}rnal of the \textbf{N}eatest and \textbf{G}reatest \textbf{L}ifeforms on \textbf{E}arth
    \end{center}
\end{frame}

\end{document}